% for Firtha, Hahn, Schultz, Fiala (2023): "Local Wave Field Synthesis by Temporal Bandlimitation", I3DA Bologna
%
% code taken from
% https://github.com/spatialaudio/wfs_chapter_hda/blob/master/graphics_DEU/spa_25d.tex
% i.e. the project wfs_chapter_hda
% - git repository https://github.com/spatialaudio/wfs_chapter_hda
% - drafts for the chapters (english, german) on **Wave Field Synthesis** for
% Stefan Weinzierl (ed.): *Handbuch der Audiotechnik*, 2nd ed., Springer,
% https://link.springer.com/referencework/10.1007/978-3-662-60357-4
% - text and graphics under CC BY 4.0 license https://creativecommons.org/licenses/by/4.0/
% - source code under MIT license https://opensource.org/licenses/MIT
% - Springer has copyright to the final english / german chapters and their layouts
% - we might also find https://git.iem.at/zotter/wfs-basics useful
% - we use violine image from https://upload.wikimedia.org/wikipedia/commons/thumb/f/f1/Violin.svg/2048px-Violin.svg.png to create picture `python/violin_wfs.png`
% - Frank Schultz, https://orcid.org/0000-0002-3010-0294, https://github.com/fs446
% - Nara Hahn, https://orcid.org/0000-0003-3564-5864, https://github.com/narahahn
% - Sascha Spors, https://orcid.org/0000-0001-7225-9992, https://github.com/spors
\documentclass[tikz]{standalone}
\usepackage{amsmath}
\usepackage{bm}
\usepackage{tkz-euclide}
\usetikzlibrary{shapes,snakes}
% we should use matplotlib colors:
\definecolor{C0}{HTML}{1f77b4}
\definecolor{C1}{HTML}{ff7f0e}
\definecolor{C7}{HTML}{7f7f7f}

% tikz stuff
% \tikzAngleOfLine from
% https://tex.stackexchange.com/questions/25342/how-to-draw-orthogonal-vectors-using-tikz
\makeatletter
\newcommand{\tikzAngleOfLine}{\tikz@AngleOfLine}
\def\tikz@AngleOfLine(#1)(#2)#3{%
  \pgfmathanglebetweenpoints{%
    \pgfpointanchor{#1}{center}}{%
    \pgfpointanchor{#2}{center}}
  \pgfmathsetmacro{#3}{\pgfmathresult}%
}
\makeatother
\tikzset{mark coordinate/.style={inner sep=0pt,
                                   outer sep=0pt,
                                   minimum size=3pt,
                                   fill=#1,
                                   circle}
                                   }

\begin{document}
\begin{tikzpicture}[scale=1,show background rectangle]
\begin{scope}

% these coordinates / parameters can be freely chosen:
\coordinate (x0) at (-1.5,0);  % determines the radius of SSD
\coordinate (xPS) at (-3.5,2);  % make sure that point source is outside of SSD
% for wave pattern 'snake=expanding waves' stuff
\def \segmentangle {30};  % opening angle of wave pattern in deg
\def \segmentlength {5};  % somehow determines the wavelength
\def \wavepatternlength {3.5};  % determines how far the wave propagates into the SSD

% ---> change with care:
\coordinate (origin) at (0,0);
% SSD circle
\node[name path=FullSSDCircle, draw, thin, color=black, circle through=(x0)] (FullSSDCircleNode) at (origin) {};
% wave pattern to x0
\draw[snake=expanding waves, segment angle=\segmentangle, segment length=\segmentlength, color=C7!33] (xPS)  -- (x0);
% wave pattern additionally into SSD, length is defined with \wavepatternlength
\tikzAngleOfLine(x0)(xPS){\angle};
\draw[snake=expanding waves, segment angle=\segmentangle, segment length=\segmentlength, color=C7!33] (xPS)  -- ++(\angle:-\wavepatternlength);
% line from point source to specific SSD point
\draw[-, thick, C7] (xPS) -- (x0) node[left, black]{$\bm{x}_0$};
% normal / tangent vectors of SSD at x0
\tikzAngleOfLine(origin)(x0){\angle};
\draw[->, >=stealth', thick, C0] (x0) -- ++(\angle:-1) node[right]{$\hat{\bm{n}}_0$};
\draw[->, >=stealth', thick, C1] (x0) -- ++(\angle-90:-1) node[below]{$\hat{\bm{n}}_t$};
% normal / tangent vectors of sound field at x0
\tikzAngleOfLine(x0)(xPS){\angle};
\draw[->, >=stealth', thick, C1] (x0) -- ++(\angle:-1) node[right]{$\hat{\bm{k}}$};
\draw[->, >=stealth', thick, C0] (x0) -- ++(\angle+90:-1) node[right]{$\hat{\bm{k}}_t$};
% black dots indicating positions of point source and specific SSD point
\draw[thick] (xPS) coordinate [mark coordinate=black, label=left:{$\bm{x}_\text{PS}$}];
\draw[thick] (x0) coordinate [mark coordinate=black];
% <---

% !!!
% these projection lines are hard coded for (x0) at (-1.5,0); (xPS) at (-3.5,2);
\draw[dashed, C0] (-0.79,0.7) -- (-0.79,0) node[below] {$\hat{k}_t$};
\draw[dashed, C1] (-0.8,-0.7) -- (-1.5,-0.7) node[left] {$\hat{k}_t$};

\end{scope}
\end{tikzpicture}
\end{document}
