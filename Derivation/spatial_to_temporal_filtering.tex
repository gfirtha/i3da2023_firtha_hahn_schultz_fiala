\documentclass[12pt,a4paper]{article}

\newcommand{\note}[1]{{\color{red}#1}}
%
\usepackage{layouts}
\usepackage{amsmath,bm}
\usepackage{a4wide}
\usepackage[T1]{fontenc}
\usepackage[utf8]{inputenc}
\usepackage{xcolor}
\usepackage{listings}
\usepackage{graphicx,overpic,subfigure}
\usepackage{tikz}
\usetikzlibrary{positioning,arrows}
\usepackage{booktabs} 			% Nice tables
\usepackage{csquotes}			% Quotation
\usepackage{multirow} 			% Multirow cells in tables
\usepackage{rotating}
\usepackage{pdflscape}
\usepackage[small,bf]{caption}
\usepackage{ae,aecompl}
\usepackage{url}
\usepackage[american]{babel}
\usepackage{hyperref}
\usepackage{nomencl}
\usepackage[toc,page]{appendix}
\usepackage{amssymb}
\usepackage{steinmetz}
\usepackage{palatino}
\usepackage{array}
\usepackage{booktabs}
\usepackage{footnote}
\usepackage{multicol}
%

\newcount\posveccount
\newcommand*\posvec[1]{
        \global\posveccount#1
        [
        \posvecnext
}
\def\posvecnext#1{
        #1
        \global\advance\posveccount-1
        \ifnum\posveccount>0
                ,\
                \expandafter\posvecnext
        \else
                ]^{\mathrm{T}}
        \fi
}

\newcount\colveccount
\newcommand*\colvec[1]{
        \global\colveccount#1
        \begin{bmatrix}
        \colvecnext
}
\def\colvecnext#1{
        #1
        \global\advance\colveccount-1
        \ifnum\colveccount>0
                \\[5pt]
                \expandafter\colvecnext
        \else
                \end{bmatrix}
        \fi
}

  
  
\newcommand{\dint}{\int\!\!\!\!\!\int}
\newcommand{\tint}{\int\!\!\!\!\int\!\!\!\!\int}
\newcommand{\qint}{\int\!\!\!\!\int\!\!\!\!\int\!\!\!\!\int}
\newcommand{\td}{\mathrm{d}}
\newcommand{\te}{\mathrm{e}}
\newcommand{\ti}{\mathrm{j}}
\newcommand{\sinfi}{\sin\varphi}
\newcommand{\cosfi}{\cos\varphi}
\newcommand{\sinteta}{\sin\theta}
\newcommand{\costeta}{\cos\theta}
\newcommand{\yref}{y_{\mathrm{ref}}}
\newcommand{\dref}{d_{\mathrm{ref}}}
\newcommand{\vx}{\mathbf{x}}
\newcommand{\vn}{\mathbf{n}}
\newcommand{\vxo}{\mathbf{x}_0}
\newcommand{\vni}{\mathbf{n}_{\mathrm{in}}}
\newcommand{\vno}{ \mathbf{n}_{\mathrm{out}} }
\newcommand{\vxs}{\mathbf{x}_{\mathrm{s}}}
\newcommand{\vxref}{\mathbf{x}_{\mathrm{ref}}}
\newcommand{\vk}{\mathbf{k}}
\newcommand{\vhk}{\hat{\mathbf{k}}}
\newcommand{\kn}{k_\mathrm{n}}
\newcommand{\kt}{k_\mathrm{t}}
\newcommand{\Oi}{\Omega_{\mathrm{i}}}
\newcommand{\Oe}{\Omega_{\mathrm{e}}}
\newcommand{\dO}{\partial \Omega}
\newcommand{\Div}{\mathrm{div}}
\newcommand{\Dx}{\nabla_{\!\!\vx}\,}
\newcommand{\Dxo}{\nabla_{\!\!\vxo}\,}
\newcommand{\Lx}{\nabla^2_{\!\!\vx}}
\newcommand{\vv}{\mathbf{v}}
\newcommand{\vvs}{\mathbf{v}_{\mathrm{s}}}
\newcommand{\mH}{\mathbf{H}}

\newcommand{\Kv}{\kappa_\mathrm{v}}
\newcommand{\Kh}{\kappa_\mathrm{h}}
\newcommand{\Rv}{\rho_\mathrm{v}}
\newcommand{\Rh}{\rho_\mathrm{h}}

\newcommand{\fgcom}[1]{{\color{red}#1}}

\newcommand{\phix}{\phi'_{x}}
\newcommand{\phixx}{\phi''_{xx}}

\newcommand{\phiy}{\phi'_{y}}
\newcommand{\phiyy}{\phi''_{yy}}

\newcommand{\phiz}{\phi'_{z}}
\newcommand{\phizz}{\phi''_{zz}}

\newcommand{\phiPxx}{\phi^{P''}_{xx}}
\newcommand{\phiGxx}{\phi^{G''}_{xx}}

\newcommand{\phiPyy}{\phi^{P''}_{yy}}
\newcommand{\phiGyy}{\phi^{G''}_{yy}}

\newcommand{\phiPzz}{\phi^{P''}_{zz}}
\newcommand{\phiGzz}{\phi^{G''}_{zz}}

\newcommand{\Phikk}{\Phi''_{k_x k_x}}

\newcommand{\tE}{t_{\mathrm{e}}}
\newcommand{\Tret}{{\scriptstyle \frac{|\vx-\vxo|}{c} }}
\newcommand{\TretS}{{\scriptstyle \frac{|\vx-\vxo^*(\vx,t)|}{c} }}

\newcommand{\FT}[2][]{
\mathcal{F}_{#1} \left\{ #2 \right\}
}
\newcommand{\IFT}[2][]{
\mathcal{F}^{-1}_{#1} \left\{ #2 \right\}
}
%
\title{Transforming spatial filters to time domain filters for sound field synthesis driving functions}
\date{\today \\
Budapest University of Technology and Economics, \\ Dept. of Networked Systems and Services, \\ Laboratory of Acoustics and Studio Technologies}
\author{Firtha Gergely}

\begin{document}
\maketitle

\subsection{General form of driving functions}

\begin{figure}[b!]
	\centering
	\begin{overpic}[width = .7\columnwidth ]{figs/general_sfs.png}
	\small
	\put(0,26){virtual source}
	\put(44.5,0.5){$\mathbf{0}$}
	\put(71,31){$\vx$}
	\put(39,21){$\vni$}
	\put(43,15){$\vxo$}
	\begin{turn}{27}
	\put(57,-3){$|\vx - \vxo|$}
	\end{turn}
	\put(50,35){$\Omega$}
	\put(80,20.5){$\dO$}
	\end{overpic}
	\caption{Geometry for the general sound field synthesis problem}
	\label{Fig:Theory:general_sfs_geometry}
\end{figure}

Assume a general 2.5D or 3D sound field synthesis (SFS) problem, as depicted for the 2.5D case in Figure \ref{Fig:Theory:general_sfs_geometry}.
The virtual source is assumed to be either a 3D plane wave or the field of a point source.
Generally, such a steady state 3D sound field can be written as
\begin{equation}
P(\omega,\vx) = A^P(\vx) \, \te^{-\ti  \omega \phi^P(\vx) } 
\end{equation}
while in the time domain reading as
\begin{equation}
    p(t,\vx) = A^P(\vx) \, \delta( t -  \phi^P(\vx) ).
\end{equation}
The phase function is related to the normalized local wavenumber vector as
\begin{equation}
    \frac{1}{c} \hat{k}^P(\vx) =  \nabla_{\vx} \, \phi^P(\vx) .
\end{equation}

The driving function that synthesises the virtual field with the secondary source distribution is given by
\begin{itemize}
    \item for a 3D SFS scenario:
     \begin{equation}
        D(\omega, \vxo) = 2 \ti \, w(\vxo) \, \frac{\omega}{c} \, \hat{\kn^P}(\vxo) \, P(\omega,\vxo) 
     \end{equation}
    \item for a 2.5D SFS scenario:
    \begin{equation}
        D(\vxo, \omega) = w(\vxo) \,
        \sqrt{ \frac{8\pi \ti \omega}{c} } \, \sqrt{\dref(\vxo)} \,
        \hat{\kn^P} (\vxo)\, P(\vxo,\omega),
     \end{equation}
\end{itemize}
In both cases the driving functions can be written in the general form of 
\begin{equation}
    D(\omega, \vxo) = A(\vxo) \, \te^{-\ti \omega \phi^P(\vxo) },
 \end{equation}
 prefiltered with a filter $H(\omega) \sim \ti \omega$ in the 3D and $H(\omega) \sim \sqrt{\ti \omega}$ in the 2.5D case, with also $A(\vx)$ depending on the dimensionality of the problem.
 Hence--similarly to the virtual field formulation-- the driving function in the time domain reads as
 \begin{equation}
   d(t, \vxo) = A(\vxo) \, \delta( t-  \phi^P(\vxo) ).
   \label{eq:general_driv_fun}
 \end{equation}
 
\subsection{Problem statement}

The spatial filtering of the driving functions is a frequently emerging question in numerous applications.
The direct application of real-time spatial filtering may, however, be computationally expensive.
This section disscusses how spatial convolutional filter can be transformed into temporal filtering by exploiting the nature of the driving functions, discussed for 2.5D synthesis case.

Assume a spatial impulse response given by $h_x(x)$.
The spatially filtered driving function is given in the time domain as 
\begin{equation}
    d'_x(t,\vxo) = \int h_x(\vxo - \vx_1) \, d(t, \vx_1) \, \td \vx_1 
    \label{eq:d_filt}
\end{equation}

Since the actual form of the driving function interconnect the spatial and temporal positions: i.e. on each spatial location the temporal shape of the driving function is identical (apart from an amplitude variation), therefore, it is possible to derive a temporal filter impulse response, performing identical filtering as given by \eqref{eq:d_filt}.
In the followings the identical temporal filter is derived

\subsection{Derivation}
Assume that a temporal impulse response $h_t(t,\vxo)$ can be formulated so that the spatially filtered driving functions are identical with the temporally filtered driving functions, i.e. 
\begin{equation}
    d'_x(t,\vxo)  = \int h_x(\vxo - \vx_1) \, d(t, \vx_1) \, \td \vx_1 = \int h_t(t - t_0,\vxo) \, d(t_0, \vxo) \, \td t_0  = d'_t(t,\vxo) 
\end{equation}
hold.
Let's substitute the assumed, simple form of the general driving functions given by \eqref{eq:general_driv_fun} to each sides resulting in
\begin{equation}
    \int h_x(\vxo - \vx_1) \, 
    A(\vx_1) \, \delta( t-  \phi^P(\vx_1) )
    \, \td \vx_1 =  A(\vxo)  \int h_t(t - t_0,\vxo)
    \, \delta( t_0 - \phi^P(\vxo) ) \, \td t_0.
\end{equation}
On the right side the sifting property of the Dirac-delta can be exploited directly:
\begin{equation}
    \int h_x(\vxo - \vx_1) \, 
    A(\vx_1) \, \delta( t-  \phi^P(\vx_1) )
    \, \td \vx_1 = A(\vxo)  h_t(t - \phi^P(\vxo) , \vxo) 
\end{equation}
i.e. the (shifted) temporal filter impulse response is given by
\begin{equation}
    h_t(t -  \phi^P(\vxo) , \vxo)  = \frac{1}{ A(\vxo) }
    \int h_x(\vxo - \vx_1) \, 
    A(\vx_1) \, \delta( t-  \phi^P(\vx_1) )
    \, \td \vx_1 
\end{equation}
For the sake of simplicity let's apply a temporal shift of $+\phi^P(\vxo)$, resulting in
\begin{equation}
    h_t(t, \vxo)  = \frac{1}{ A(\vxo) }
    \int h_x(\vxo - \vx_1) \, 
    A(\vx_1) \, \delta( t + \phi^P(\vxo)- \phi^P(\vx_1) )
    \, \td \vx_1 
\end{equation}

\vspace{1cm}
The spatial convolution can be evaluated by exploiting the generalized sifting property of the Dirac-delta
\begin{equation}
    \int_{-\infty}^{\infty} f(x) \, \delta(g(x)) \, \td x = \sum_{i} \frac{f(x_i)}{\left| \frac{\partial}{\partial x} g(x) \right|_{x = x_i}}, \hspace{5mm} \text{where} \hspace{5mm} g(x_i) = 0.
\end{equation}
or in terms of contour integration as
\begin{equation}
    \int f(\vx_1) \, \delta(g(\vx_1)) \, \td \vx_1 = \sum_{i} \frac{f(\vx_i)}{\left| \nabla_{\vx_1} \, g(\vx_1) \cdot \vv(\vx_1) \right|_{\vx_1 = \vx_i}}, \hspace{5mm} \text{where} \hspace{5mm} g(\vx_i) = 0.
\end{equation}
In the current scenario 
\begin{align}
    g(\vx_1) &= t + \phi^P(\vxo)- \phi^P(\vx_1) \\
    \nabla_{\vx_1} g(\vx_1) &= - \frac{1}{c}  \hat{\vk}^P(\vx_1) \\
    f(\vx_1)  &= h_x(\vxo - \vx_1) \,  A(\vx_1)
\end{align}
For sake of simplicity it is assumed that a single zero position for the Dirac-argument exists, given where
\begin{equation}
     \phi^P(\vx_i) = t + \phi^P(\vxo)
\end{equation}
is satisfied. 
The physical interpretation of this assumption is discussed later. 

By denoting the time dependent zero-position as $\vx_i(t)$ the integral simplifies to 
\begin{equation}
    h_t(t, \vxo)  = \frac{A(\vx_i(t)) }{ A(\vxo) }  \,
    \frac{c}{| \hat{k_\mathrm{v}}^P(\vx_i(t))  |} \,
     h_x(\vxo - \vx_i(t))
\end{equation}
with $\hat{k_\mathrm{v}}^P$ denoting the tangential component of the normalized (time independent) local wavenumber vector

\subsection{Linear SSD, plane wave}
For a plane wave the phase function is given as
\begin{equation}
     \phi^P(\vx) = \frac{1}{c} (\hat{k_x} x + \hat{k_y} y),
\end{equation}
furthermore, let's assume a linear SSD with direction vector $\vv = [v_x, \, v_y]$, hence both $\vxo$ and $\vx_i$ satisfy
\begin{align}
    0 &= -v_y x_0 + v_x y_0 + c,\\
    0 &= -v_y  x_i + v_x y_i + c,\\
    0 &= -v_y (x_i-x_0) + v_x (y_i-y_0),\\
    y_i-y_0 &= \frac{v_y}{v_x} (x_i-x_0)\\
    x_i-x_0 &= \frac{v_x}{v_y} (y_i-y_0).
\end{align}
Hence, indeed, for a plane wave arriving at a linear SSD a single solution exists for the above system of equation as long as the plane wave does not arrive normally to the SSD (this latter case will be further examined later):
For a given $\vxo$, position $\vx_i$ is found, where
\begin{equation}
    \hat{k_x} x_i + \hat{k_y} y_i = t c + \hat{k_x} x_0 + \hat{k_y} y_0
\end{equation}
\begin{equation}
    \hat{k_x} (x_i-x_0) + \hat{k_y} (y_i-y_0) = t c
\end{equation}

\begin{equation}
    x_i = x_0 + v_x\frac{t c}{\hat{k_x}v_x+ \hat{k_y} v_y } 
\end{equation}
\begin{equation}
    y_i = y_0 + v_y\frac{t c}{\hat{k_x} v_x  + \hat{k_y}v_y} 
\end{equation}
It can be realized that $\hat{k_x} v_x  + \hat{k_y}v_y = \vv \cdot \hat{\vk^P} = \hat{k_v^P}$, i.e. by vector notation
\begin{equation}
    \vx_i = \vxo + \vv \frac{t c}{\hat{k_v^P}} 
\end{equation}
And the filter impulse response reads as
\begin{equation}
    h_t(t, \vxo)  = \frac{A( \vxo + \vv \frac{t c}{\hat{k_v^P}}) }{ A(\vxo) }  \,
    \frac{c}{| \hat{k_\mathrm{v}}^P( \vxo + \vv \frac{t c}{\hat{k_v^P}})  |} \,
     h_x(-\vv \frac{t c}{\hat{k_v^P}})
\end{equation}
In the followings the following assumptions are made:
\begin{itemize}
    \item it is assumed that the spatial filter impulse response is relatively short, and the amplitude factor of the driving function changes slowly in the proximity of $\vxo$, i.e.
$A( \vxo + \vv \frac{t c}{\hat{k_v^P}}) \approx A(\vxo)$.
    \item it is assumed that the incident field is locally plane at $\vxo$, i.e. again, in the proximity of $\vxo$ $\hat{k_\mathrm{v}}^P( \vxo + \vv \frac{t c}{\hat{k_v^P}}) \approx  \hat{k_\mathrm{v}}^P( \vxo)$ holds.
\end{itemize}
Similarly to standard WFS assumptions, the assumptions hold for SSD elements with small curvature and for virtual field with small local curvature (plane waves and far-field of point sources). 
Finally it is exploited that usually the spatial filter impulse reponses are defined as the function of a single parameter, i.e. the vector notation in the function's argument is redunant.
With all these considerations the connection between the spatial filter and the corresponding temporal filter at a single SSD element at $\vxo$ is given by 
\begin{equation}
    h_t(t, \vxo)  = \frac{c}{\left| \hat{k_\mathrm{v}^P}( \vxo ) \right| } \,
     h_x(-t \frac{c}{\hat{k_v^P}(\vxo)}).
\end{equation}
Hence, the temporal impulse response is obtained by rescaling the spatial impulse response.

\vspace{3mm}
The above formulation allows the interconnection of the spatial filter transfer function with the corresponding temporal filter frequency characteristics as well.

Assume that the transfer function of the spatial filter is given by $H_x(k_x)$.
The corresponding temporal filter transfer is obtained simply from the scaling property of the Fourier transformed
\begin{equation}
    \frac{1}{2\pi}\int h_t(t, \vxo) \te^{-\ti \omega t} \td t  = \frac{c}{\left| \hat{k_\mathrm{v}^P}( \vxo ) \right| } \frac{1}{2\pi} \int 
     h_x(-t \frac{c}{\hat{k_v^P}(\vxo)}) \te^{-\ti \omega t} \td t
\end{equation}
\begin{equation}
 H_t(\omega, \vxo) =  H_x( -\omega\frac{\hat{k_v^P}(\vxo)}{c}) 
\end{equation}


\subsection{Alternative derivation: stationary position for the wavenumber domain driving functions}
See: Firtha PhD, Appendix D.

\subsection{Application example: antialiasing filtering}

\end{document}